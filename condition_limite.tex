\subsection{Conditions aux bords}


%---------------------------
\begin{frame}
\frametitle{Dérivation Spatiale}

On utilise la dérivée dite \textit{amont} : 

\begin{equation*}
\left. \frac{\partial \nu^{\star} S^{\star}}{\partial x} \right|_i = \frac{\nu^{\star}S^{\star}(i+1)-\nu^{\star}S^{\star}(i)}{\delta x} 
\end{equation*}


Quand $i=i_{max} \rightarrow \nu^{\star}S^{\star}(i_{max}+1) $??

\end{frame}
%---------------------------


%---------------------------
\begin{frame}
\frametitle{Grandeurs à dériver}
Dérivées spatiales intervenant :

\begin{itemize}
    \item dérivée première :
    \begin{itemize}
        \item $T^*$
        \item $ \nu^{\star}S^{\star}$
        \item $\frac{S^{\star}}{x}$
    \end{itemize}
    \item dérivée seconde :
    \begin{itemize}
        \item $ \nu^{\star}S^{\star}$
    \end{itemize}
\end{itemize}

\end{frame}
%---------------------------


%---------------------------
\begin{frame}
\frametitle{Conditions aux bords}
\framesubtitle{$T^*$ ; $\nu^{\star} S^{\star}$}

\begin{itemize}

\onslide<1->
\item $\left.\frac{\partial T^{\star}}{\partial x}\right|_{i_{max}} =0$ par condition sur notre modèle

\onslide<2->
\item $\dot{M}^{\star} = \frac{\partial \nu^{\star} S^{\star}}{\partial x}$, donc $ \dot{M_0}^{\star} = \left. \frac{\partial \nu^{\star} S^{\star}}{\partial x} \right|_{i_{max}} $

\end{itemize}

\end{frame}
%---------------------------

%---------------------------
\begin{frame}
\frametitle{Conditions aux bords}
\framesubtitle{$\nu^{\star} S^{\star}$}

\only<1-3>{
$$\left. \frac{\partial \nu^{\star} S^{\star}}{\partial x} \right|_{i_{max}} = \dot{M_0}^{\star} = \frac{\nu^{\star}S^{\star}(i_{max}+1)-\nu^{\star}S^{\star}(i_{max})}{\delta x} $$ 
}
\only<2-3>{
On en déduit 

$$\nu^{\star}S^{\star}(i_{max}+1) = \dot{M_0}^{\star}\ \delta x + \nu^{\star}S^{\star}(i_{max})$$ 
}

\only<3>{
Donc 
$$  \left. \frac{\partial^2 (\nu^{\star} S^{\star})}{\partial x^2}\right|_{i=i_{max}} = \frac{\dot{M_0}^{\star}\ \delta x - (\nu^{\star} S^{\star})(i_{max}) + (\nu^{\star} S^{\star})(i_{max}-1)}{\delta x^2}$$
}

\only<4>{
en $i=0$ on a  $ \nu^{\star} S^{\star} =0$  d'où 
$$ \left. \frac{\partial^2 \nu^{\star}S^{\star}}{\partial x^2}\right|_{i=1}=\frac{(\nu^{\star}S^{\star})(1+1)-2\ (\nu^{\star}S^{\star})(1)}{\delta x^2}$$
 }
\end{frame}
%---------------------------

%---------------------------
\begin{frame}
\frametitle{Conditions aux bords}
\framesubtitle{$S^*/x$}

En $i_{max}$ 
$$Q_{adv}=\frac{\mathcal{R} T_0}{\mu} \frac{4-3\beta}{\beta} \frac{T^\star}{S^\star} \times
        \left( \frac{\partial S^\star}{\partial t^\star} + v^\star \frac{\partial}{\partial x} \left(\frac{S^\star}{x}\right) \right) -
        \frac{C_v^\star v^\star}{x} \frac{\partial T^\star}{\partial x}=0 $$

Donc
$$  \frac{\partial}{\partial x} \left(\frac{S^{\star}}{x}\right) = - \frac{1}{v^{\star}} \frac{\partial S^{\star}}{\partial t^{\star}} $$

$$ \left. \frac{\partial}{\partial x} \left(\frac{S^{\star}}{x}\right) \right|_{i_{max}} = \left. -\frac{1}{x^2\ v^{\star}} \frac{\partial^2 (\nu^{\star} S^{\star})}{\partial x^2} \right|_{i_{max}} $$
\end{frame}
%---------------------------