\section{Les paramètres du modèle}
\begin{frame}
\frametitle{Les paramètres d'entrée de la simulation}
\begin{itemize}
	\item 6 paramètres d'entrée :
	\begin{itemize}
		\item $M$ : masse du trou noir
		\item $r_{max}$ = rayon maximal du trou noir
		\item $X$ : fraction en masse d'hydrogène
		\item $Y$ : fraction en masse d'hélium
		\item $\dot{M}_{0}$ : taux d'accrétion au bord du disque
		\item $\alpha$ : paramètre phénoménologique libre 
	\end{itemize}
\end{itemize}
\end{frame} 

\begin{frame}
\frametitle{Considérations sur la viscosité}
	\begin{itemize}
		\item Ignorance sur les mécanismes de viscosité isolée dans 					$\alpha$ \\
		
		\item Limite du paramètre $\alpha$ : Shakura and Sunyaev (1973)
			\begin{itemize}
				\item $\alpha \le 1$
				\item Approche semi empirique au problème de viscosité
				\item Turbulence $\rightarrow$ $\lambda_{turb}$, 							$v_{turb}$
				\item $\nu_{turb} \sim \lambda_{turb} v_{turb}$
				\item $\lambda_{turb} \le H$
				\item Improbable que $v_{turb} > c_{s}$
			\end{itemize}
		\item $\nu = \alpha c_{s} H$
	\end{itemize}
\end{frame}